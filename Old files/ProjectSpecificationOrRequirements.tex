\section{Project specification/Requirements}

\subsection{Project description}
Advanced Driver Assistance Systems (ADAS) have become an integral part of modern high-end cars. With data from cameras and other sensors, they perform services such as pedestrian detection, blind spot detection, and lane departure warning. The computational requirements for many of the algorithms needed to process the data are substantial. To have price, size, compute and energy efficient solutions, a combination of hardware and software is typically required.  NTNU is partner in the EU research project Tulipp (Towards Ubiquitous Low-power Image Processing Platforms), where a system for pedestrian detection is one of the use-cases. A software implementation of the system is available in an open source repository.\\

\noindent
In this project assignment, the pedestrian detection system shall be investigated from a hardware/software codesign perspective. Based on a literature study and early investigation of the system code, a design methodology shall be selected and described. Following the selected methodology, critical parts of the system shall be implemented in hardware using design tools suitable for Xilinx FPGAs. Overall improvement in performance and energy efficiency compared to an all software solution shall be estimated. To the extent time allows, test on an FPGA evaluation board shall be performed. \\

\begin{itemize}
    \item Develop HW/SW Codesign methodology
    \item Estimate overall improvement
\end{itemize}