\chapter{Introductory part}
\section{Introduction and motivation}
%\addcontentsline{toc}{section}{\numberline{}Introduction and motivation}
Electronic gadgets of different shapes and sizes are a vital part of today's society. With ever increasing demands from the market for these gadgets, that require both increased performance as well as more advanced functionality, the need to increase the efficiency in the design process for these gadgets has become increasingly important for developers and producers. This efficiency increase is improtant to enable them to keep their position in the tough and highly competitive market. To accomodate the demands, several techniques for increasing the efficiency has been developed, and one particularly well regarded set of techniques is called Hardware/Software Co-design, or HW/SW Codesign for short. HW/SW Codesign is a design philosophy developed for the purpose of improving the cooperation between hardware and software of electronic applications, and utilizes the synergy between hardware and software to achieve this. 
\\\\
\noindent
The HW/SW Codesign philosophy is carried out by focusing on the concurrent design of hardware and software, in order to meet the given system-level objectives.\cite{ReadingsHWSW} 
With digital systems often being developed by several organizations and for different applications, the cooperation between the hardware and software parts within the system of these applications is not always considered optimal, which gives rise to a motivation for using the HW/SW Codesign philosophy. Notable HW/SW Codesign methods involve using digital HW design tools to design the hardware, as well as using simulation tools for testing the hardware in the design phase instead of the production phase. 
\\\\


\begin{comment}
This report will describe a process that has been carried out to select an appropriate HW/SW Codesign methodology for certain applications of an existing Pedestrian Detection system. The report will follow a traditional report structure. It starts with an introductory part, which will be followed by a main part in the middle, and finally a concluding part at the end. At the bottom of the report there will be an appendix containing all graphs, figures, tables etc that were deemed too comprehensive or large in size to be included in the main parts of the report. All measurements and results of the project will be described in the main part, while the concluding part will elaborate, discuss and conclude further on these topics, as well as propose recommendations for future work. 
\\\\\\\\
\end{comment}



\noindent
\textit{\color{red}\Big{Introductionandmotivation:}\\
This is normally Chapter 1 and gives an overview of the assignment and why this work is important. If you have chosen to focus mainly on a part of the assignment text, you may write something about this here and explain why. You also normally give a short description of the structure of the rest of the report towards the end of this chapter.  It is important to indicate which parts that are based on your own work. You may even include a list of your main contributions. }\\



\clearpage