\section*{Preface}
\addcontentsline{toc}{section}{\numberline{}Preface}
This specialization project has delivered a great learning outcome in terms of practical experience with previously acquired theoretical knowledge. The HW/SW Codesign topic was chosen partly due to its relevancy to my study programme specialization, but also because of the aforementioned practical learning outcome I assumed it would give me. The project has given me a deeper insight into the topic of HW/SW Codesign, as I expected it to.  \\
\noindent

\begin{comment}
Despite the learning outcome, the task accomplishment is less satisfactory than anticipated in terms of fulfillment of the project goals. The work process has often been slowed down due to specific tasks requiring more time to accomplish than anticipated, or due to encountering several unexpected problems in several areas related to the task that needed to be solved. Despite this, it is my belief that the accomplished results can be considered good enough in their own right.\\\\
\end{comment}

\noindent
I would like to thank my supervisor Per Gunnar Kjeldsberg for continually providing me with council on which steps to take forward project wise, as well as for giving feedback on the report at several instances during the project phase. \\\\\\\\


\noindent
\textit{\color{red}\Big{Preface:}\\
Here you can describe the process that led to the writing of this report (i.e., the project work process). Here you can mention why you chose this assignment and specific challenges you met with during your work. You can also thank people for help and support during this project.}\\
\\
\noindent
\subsection{Overview of report}
This report will describe a process that has been carried out to select an appropriate HW/SW Codesign methodology for certain applications of an existing Pedestrian Detection system. The report will start with an introductory part, which will be followed by a main  part in the middle, and finally a concluding part at the end. 
\\
\noindent
\subsubsection{Part 1: Introductory part}
The introductory part will contain the assignment text, as well as the abstract, preface and table of contents, in addition to this current chapter with introduction and motivation. The objective of the introductory part as a whole is to introduce the reader to the HW/SW Codesign philosophy, as well as to give a description of the project and the work process that has led to the finished report. 
\\
\noindent
\subsubsection{Part 2: Main part}
The main part begins with a chapter about the theory and background of HW/SW Codesign, which will describe the topic in theoretical detail. The part further continues with a chapter about some of the work that has been done on the topic in the past, before providing some information about the Pedestrian Detection system that is the basis for the work in this project. Following from this comes the measurements and results from the project work, which will contain graphs, tables and figures derived from the project work. 
Finally, the main part will feature a presentation of the chosen HW/SW Codesign methodology that is the result of the project work as a whole. 
\\
\noindent
\subsubsection{Part 3: Concluding part}
The concluding part will feature a chapter for discussions and one for conclusions, both in relation to the chosen methodology. It will thus debate whether or not the results are conclusive enough and if the methodology could have been improved in any way. 
\\
\noindent
\subsubsection{Part 4: Appendix}
At the bottom of the report there will be an appendix containing all graphs, figures, tables etc that were deemed too comprehensive or large in size to be included in the main parts of the report. 


\clearpage


